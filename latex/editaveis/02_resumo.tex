\begin{resumo}
%\textcolor{red}{O resumo é um item obrigatório. Essa parte do relatório será uma visão rápida e clara do projeto desenvolvido. O leitor terá informações como: descrição breve do projeto, principais requisitos, tecnologias necessárias e outras informações relevantes para apresentação. O resumo terá no máximo meia (1/2) página.}

%\textcolor{red}{Ao longo deste texto, as descrições em \textbf{cor vermelha} são meras instruções, não devendo aparecer na versão final do texto. \textbf{Utilize o comentário \% ao invés de apagar estas descrições, para não perder as orientações apresentadas.}}

O projeto Controle de Trajetória de Foguetes d’Água foi desenvolvido no âmbito do Projeto Integrador de Engenharia 1 da Faculdade UnB Gama, com a colaboração de estudantes das engenharias de Software e Aeroespacial. O objetivo principal é construir um foguete reutilizável que utilize água como combustível e atinja distâncias fixas de 10, 20 e 30 metros com precisão de ±0,5 metro, além de uma plataforma de lançamento automatizada que garanta segurança operacional.

O sistema incorpora um conjunto de sensores para medição em tempo real de parâmetros como pressão, ângulo de lançamento, velocidade e altitude, integrados a um microcontrolador ESP32 para processamento e transmissão de dados. A solução prevê ainda a persistência das informações em um banco de dados para calibração e análise de trajetórias. Entre os principais requisitos, destacam-se a automação eletromecânica do lançamento, a reutilização do foguete em três missões e o cumprimento de normas de segurança, como distanciamento mínimo de 5 metros de pessoas.

\vspace{\onelineskip}
\noindent
\textbf{Palavras-chaves}:{ \imprimirpalavrachaveum, \imprimirpalavrachavedois, \imprimirpalavrachavetres, \imprimirpalavrachavequatro, \imprimirpalavrachavecinco.}
\end{resumo}

% \begin{resumo}
%  O resumo deve ressaltar o objetivo, o método, os resultados e as conclusões 
%  do documento. A ordem e a extensão
%  destes itens dependem do tipo de resumo (informativo ou indicativo) e do
%  tratamento que cada item recebe no documento original. O resumo deve ser
%  precedido da referência do documento, com exceção do resumo inserido no
%  próprio documento. (\ldots) As palavras-chave devem figurar logo abaixo do
%  resumo, antecedidas da expressão Palavras-chave:, separadas entre si por
%  ponto e finalizadas também por ponto. O texto pode conter no mínimo 150 e 
%  no máximo 500 palavras, é aconselhável que sejam utilizadas 200 palavras. 
%  E não se separa o texto do resumo em parágrafos.

%  \vspace{\onelineskip}
    
%  \noindent
%  \textbf{Palavras-chave}: latex. abntex. editoração de texto.
% \end{resumo}
