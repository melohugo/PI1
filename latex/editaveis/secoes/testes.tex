\subsection{Plano de testes}

\begin{figure}[H]
    \centering
\begin{longtable}{|p{0.2\textwidth}|p{0.7\textwidth}|}
\hline
\multicolumn{2}{|l|}{\textbf{CT01 - Importação de Arquivo JSON Válido}} \\
\hline
\textbf{Tipo} & Teste de sistema \\
\hline
\textbf{Objetivo} & Verificar se o sistema é capaz de importar corretamente um arquivo JSON com dados válidos. \\
\hline
\textbf{Pré-condições} & O sistema deve estar iniciado. Deve haver um arquivo "database.json" com dados completos e corretos no formato esperado.  \\
\hline
\textbf{Procedimentos} & Abrir o sistema. Acessar a funcionalidade "Importar Dados". Selecionar o arquivo "database.json". Confirmar importação. \\
\hline
\textbf{Resultado esperado} & O sistema exibe mensagem de sucesso. Os dados são carregados na interface. Nenhum erro é exibido. \\
\hline
\textbf{Especificação do Reparo} & Verificar se o parser JSON está tratando corretamente os campos esperados. Validar se o caminho de acesso ao arquivo está correto. Corrigir o tratamento de erros silenciosos na importação. \\
\hline
\textbf{Resultado Após Reparo} & O sistema importa corretamente arquivos válidos e exibe os dados esperados. Teste é reexecutado e aprovado. \\
\hline
\end{longtable}
\caption{Caso de Teste 01 - Importação de Arquivo JSON Válido}
\label{fig_ct01_importacao_json_valido}
\end{figure}

\begin{figure}[H]
    \centering
\begin{longtable}{|p{0.2\textwidth}|p{0.7\textwidth}|}
\hline
\multicolumn{2}{|l|}{\textbf{CT02 - Importação de JSON com campos ausentes}} \\
\hline
\textbf{Tipo} & Sistema \\
\hline
\textbf{Objetivo} & Verificar o comportamento do sistema ao importar registros incompletos. \\
\hline
\textbf{Pré-condições} & Arquivo "dados\_incompletos.json" com campos obrigatórios ausentes. \\
\hline
\textbf{Procedimentos} & Acessar "Importar Dados". Selecionar "dados\_incompletos.json". Confirmar importação. \\
\hline
\textbf{Resultado esperado} & Sistema identifica e ignora os registros inválidos. \\
\hline
\textbf{Especificação do Reparo} & Adicionar validação de campos obrigatórios durante o parsing. \\
\hline
\textbf{Resultado após Reparo} & Registros incompletos tratados corretamente. \\
\hline
\end{longtable}
\caption{Caso de Teste 02 - Importação de JSON com campos ausentes}
\label{fig_ct02_importacao_json_campos_ausentes}
\end{figure}

\begin{figure}[H]
    \centering
\begin{longtable}{|p{0.2\textwidth}|p{0.7\textwidth}|}
\hline
\multicolumn{2}{|l|}{\textbf{CT03 - Importação de formato inválido}} \\
\hline
\textbf{Tipo} & Sistema \\
\hline
\textbf{Objetivo} & Garantir que o sistema rejeita arquivos com extensão JSON mas conteúdo inválido (ex: JSON). \\
\hline
\textbf{Pré-condições} & Arquivo "dados\_errados.json" contendo dados JSON. \\
\hline
\textbf{Procedimentos} & Tentar importar "dados\_errados.json". \\
\hline
\textbf{Resultado esperado} & Mensagem de erro de formato exibida. \\
\hline
\textbf{Especificação do Reparo} & Validar estrutura JSON no momento da importação. \\
\hline
\textbf{Resultado após Reparo} & Sistema rejeita corretamente arquivos malformados. \\
\hline
\end{longtable}
\caption{Caso de Teste 03 - Importação de formato inválido}
\label{fig_ct03_importacao_formato_invalido}
\end{figure}

\begin{figure}[H]
    \centering
\begin{longtable}{|p{0.2\textwidth}|p{0.7\textwidth}|}
\hline
\multicolumn{2}{|l|}{\textbf{CT04 - Exportação de Dados para JSON}} \\
\hline
\textbf{Tipo} & Sistema \\
\hline
\textbf{Objetivo} & Garantir que os dados processados sejam exportados corretamente para JSON. \\
\hline
\textbf{Pré-condições} & Dados processados e disponíveis na interface. \\
\hline
\textbf{Procedimentos} & Clicar em "Exportar Dados".  Selecionar "Formato JSON".  Salvar arquivo.  \\
\hline
\textbf{Resultado esperado} & Arquivo JSON é gerado com os dados exibidos. \\
\hline
\textbf{Especificação do Reparo} & Corrigir função de exportação e formatação de colunas. \\
\hline
\textbf{Resultado após Reparo} & Arquivo JSON correto é gerado. \\
\hline
\end{longtable}
\caption{Caso de Teste 04 - Exportação de Dados para JSON}
\label{fig_ct05_exportacao_dados_csv}
\end{figure}

\begin{figure}[H]
    \centering
\begin{longtable}{|p{0.2\textwidth}|p{0.7\textwidth}|}
\hline
\multicolumn{2}{|l|}{\textbf{CT05 – Responsividade da Interface}} \\
\hline
\textbf{Tipo} & Sistema \\
\hline
\textbf{Objetivo} & Verificar o comportamento da interface em diferentes resoluções da tela. \\
\hline
\textbf{Pré-condições} & Sistema em execução. \\
\hline
\textbf{Procedimentos} & Acessar a aplicação em diferentes tamanhos de tela. Navegar pelas funcionalidades. \\
\hline
\textbf{Resultado esperado} & Componentes se ajustam corretamente, sem sobreposições ou corte. \\
\hline
\textbf{Especificação do Reparo} & Ajustar responsividade. \\
\hline
\textbf{Resultado após Reparo} & Interface se adapta corretamente a todas as resoluções testadas. \\
\hline
\end{longtable}
\caption{Caso de Teste 05 - Responsividade da Interface}
\label{fig_ct06_responsividade_interface}
\end{figure}

\begin{figure}[H]
    \centering
\begin{longtable}{|p{0.2\textwidth}|p{0.7\textwidth}|}
\hline
\multicolumn{2}{|l|}{\textbf{CT06 - Validação dos sensores no lançamento do foguete}} \\
\hline
\textbf{Tipo} & Sistema \\
\hline
\textbf{Objetivo} & Verificar se os dados coletados pelos sensores (pressão, ângulo, massa) estão sendo corretamente lidos e registrados no início do lançamento. \\
\hline
\textbf{Pré-condições} & Sistema embarcado ligado.  Foguete pronto para lançamento.  Todos os sensores conectados corretamente. \\
\hline
\textbf{Procedimentos} & Ligar o sistema de aquisição de dados.  Acionar o lançamento do foguete.  Verificar os dados registrados pelos sensores na interface de análise. \\
\hline
\textbf{Resultado Esperado} & Os dados de pressão, ângulo e massa devem ser capturados corretamente, sem valores nulos ou inconsistentes, e armazenados no JSON gerado. \\
\hline
\textbf{Especificação do Reparo} & Verificar conexões físicas dos sensores, calibrar sensores com valores reais, corrigir erros de leitura no firmware. \\
\hline
\textbf{Resultado Após Reparo} & Os sensores coletam os dados de forma correta e os valores são exibidos e armazenados conforme esperado. \\
\hline
\end{longtable}
\caption{Caso de Teste 06 - Validação dos sensores no lançamento do foguete}
\label{fig_ct07_validacao_sensores_lancamento_foguete}
\end{figure}

\begin{figure}[H]
    \centering
\begin{longtable}{|p{0.2\textwidth}|p{0.7\textwidth}|}
\hline
\multicolumn{2}{|l|}{\textbf{CT07 - Funcionamento do mecanismo de lançamento}} \\
\hline
\textbf{Tipo} & Sistema \\
\hline
\textbf{Objetivo} & Verificar o funcionamento do mecanismo de lançamento. \\
\hline
\textbf{Pré-condições} & Sistema embarcado ligado.  Foguete carregado com água e pressurizado. \\
\hline
\textbf{Procedimentos} & Acionar o sistema de lançamento via interface.  Observar o funcionamento do atuador eletromecânico. \\
\hline
\textbf{Resultado Esperado} & O foguete deve ser lançado automaticamente após o comando, sem falhas mecânicas ou atraso. \\
\hline
\textbf{Especificação do Reparo} & Verificar fiação, tensão e código do acionamento do atuador. \\
\hline
\textbf{Resultado Após Reparo} & Atuador funciona corretamente e o foguete é lançado de forma imediata e estável. \\
\hline
\end{longtable}
\caption{Caso de Teste 07 - Funcionamento do mecanismo de lançamento}
\label{fig_ct08_funcionamento_mecanismo_lancamento}
\end{figure}

\subsection*{Testes de Integração}

\begin{figure}[H]
    \centering
\begin{longtable}{|p{0.2\textwidth}|p{0.7\textwidth}|}
\hline
\multicolumn{2}{|l|}{\textbf{CT08 - Geração dos gráficos}} \\
\hline
\textbf{Tipo} & Integração \\
\hline
\textbf{Objetivo} & Verificar se o sistema gera corretamente o gráfico de altitude após a importação dos dados. \\
\hline
\textbf{Pré-condições} & Dados válidos já importados. \\
\hline
\textbf{Procedimentos} & Acessar a interface de visualização de gráficos. \\
\hline
\textbf{Resultado Esperado} & Gráfico exibido com dados coerentes. \\
\hline
\textbf{Especificação do Reparo} & Ajustar lógica de plotagem ou eixos do gráfico. \\
\hline
\textbf{Resultado após Reparo} & Gráfico é exibido corretamente. \\
\hline
\end{longtable}
\caption{Caso de Teste 08 - Geração dos gráficos}
\label{fig_ct04_geracao_graficos}
\end{figure}

\begin{figure}[H]
    \centering
\begin{longtable}{|p{0.2\textwidth}|p{0.7\textwidth}|}
\hline
\multicolumn{2}{|l|}{\textbf{CT09 - Integração entre ESP32 e Sensores}} \\
\hline
\textbf{Tipo} & Integração \\
\hline
\textbf{Objetivo} & Verificar se os sensores (pressão, ângulo, massa) estão integrados corretamente ao ESP32 e geram dados coerentes. \\
\hline
\textbf{Pré-condições} & Firmware embarcado finalizado.  Todos os sensores ligados ao ESP32. \\
\hline
\textbf{Procedimentos} & Iniciar sistema embarcado.  Coletar dados em tempo real dos sensores.  Verificar se os dados são salvos e transmitidos corretamente. \\
\hline
\textbf{Resultado esperado} & Leituras coerentes e disponíveis para transmissão e armazenamento. \\
\hline
\textbf{Especificação do Reparo} & Verificar drivers de sensores, conexões físicas e timing de leitura. \\
\hline
\textbf{Resultado após Reparo} & Sistema lê e integra dados sem falhas ou atrasos. \\
\hline
\end{longtable}
\caption{Caso de Teste 09 - Integração entre ESP32 e Sensores}
\label{fig_ct24_integracao_esp32_sensores}
\end{figure}

\begin{figure}[H]
    \centering
\begin{longtable}{|p{0.2\textwidth}|p{0.7\textwidth}|}
\hline
\multicolumn{2}{|l|}{\textbf{CT10 - Integração Energética – Estabilidade no Sistema Completo}} \\
\hline
\textbf{Tipo} & Integração \\
\hline
\textbf{Objetivo} & Verificar se a fonte de energia dimensionada suporta o consumo de todos os componentes ao mesmo tempo. \\
\hline
\textbf{Pré-condições} & Sistema montado com todos os sensores e atuadores ativos.  Fonte de energia de 3,3 V e corrente maior que 300\,mA. \\
\hline
\textbf{Procedimentos} & Ligar todos os módulos simultaneamente (ex: pressão, atuadores, display, etc.).  Observar estabilidade de tensão e corrente durante 30 segundos.  Verificar se não ocorrem quedas ou falhas. \\
\hline
\textbf{Resultado esperado} & Tensão estável ($\pm 6\%$) e corrente $\le$ capacidade máxima ($300$)mA. \\
\hline
\textbf{Especificação do Reparo} & Substituir a fonte por uma de maior capacidade ou revisar fiação. \\
\hline
\textbf{Resultado após Reparo} & Sistema opera sem oscilações ou falhas de alimentação. \\
\hline
\end{longtable}
\caption{Caso de Teste 10 - Integração Energética – Estabilidade no Sistema Completo}
\label{fig_ct26_integracao_energetica_estabilidade_sistema_completo}
\end{figure}

\subsection*{Testes de Unidade}

\begin{figure}[H]
    \centering
\begin{longtable}{|p{0.2\textwidth}|p{0.7\textwidth}|}
\hline
\multicolumn{2}{|l|}{\textbf{CT11 - Parser de JSON Válido}} \\
\hline
\textbf{Tipo} & Unidade \\
\hline
\textbf{Objetivo} & Verificar leitura de JSON com dados completos. \\
\hline
\textbf{Pré-condições} & Função de parser implementada e disponível no ambiente de desenvolvimento.  Arquivo de teste válido com dados completos. \\
\hline
\textbf{Procedimentos} & Rodar função de parser diretamente com arquivo de teste válido.  Verificar se estrutura retornada está correta. \\
\hline
\textbf{Resultado esperado} & Dados carregados na memória, sem exceções. \\
\hline
\textbf{Especificação do Reparo} & Revisar lógica de leitura e formatação do JSON.  Corrigir tratamento de exceções ou formatação incorreta. \\
\hline
\textbf{Resultado após Reparo} & Parser reconhece corretamente arquivos válidos e gera a estrutura esperada sem exceções. \\
\hline
\end{longtable}
\caption{Caso de Teste 11 - Parser de JSON Válido}
\label{fig_ct09_parser_json_valido}
\end{figure}

\begin{figure}[H]
    \centering
\begin{longtable}{|p{0.2\textwidth}|p{0.7\textwidth}|}
\hline
\multicolumn{2}{|l|}{\textbf{CT12 - Parser de JSON com Campos Ausentes}} \\
\hline
\textbf{Tipo} & Unidade \\
\hline
\textbf{Objetivo} & Garantir tratamento de dados incompletos. \\
\hline
\textbf{Pré-condições} & Função de parser em funcionamento.  JSON de teste com campos ausentes. \\
\hline
\textbf{Procedimentos} &  Rodar parser com JSON que tenha campos ausentes.  Verificar se registros inválidos são removidos. \\
\hline
\textbf{Resultado esperado} & Retorno apenas dos registros completos. \\
\hline
\textbf{Especificação do Reparo} & Corrigir verificação de campos obrigatórios.  Implementar descartes controlados de registros inválidos. \\
\hline
\textbf{Resultado após Reparo} & Parser filtra registros incompletos sem falhas. \\
\hline
\end{longtable}
\caption{Caso de Teste 12 - Parser de JSON com Campos Ausentes}
\label{fig_ct10_parser_json_campos_ausentes}
\end{figure}

\begin{figure}[H]
    \centering
\begin{longtable}{|p{0.2\textwidth}|p{0.7\textwidth}|}
\hline
\multicolumn{2}{|l|}{\textbf{CT13 - Parser de JSON com Dados Absurdos}} \\
\hline
\textbf{Tipo} & Unidade \\
\hline
\textbf{Objetivo} & Verificar descarte de dados absurdos (ex: altitude negativa). \\
\hline
\textbf{Pré-condições} & Função de validação em funcionamento.  Lista de dados brutos com registros absurdos. \\
\hline
\textbf{Procedimentos} & Rodar função de validação com lista de dados brutos.  Observar se registros absurdos são filtrados. \\
\hline
\textbf{Resultado esperado} & Lista resultante sem registros absurdos. \\
\hline
\textbf{Especificação do Reparo} & Corrigir regra de validação de campos.  Implementar descarte de dados absurdos. \\
\hline
\textbf{Resultado após Reparo} & Função remove corretamente dados absurdos sem impactar o restante. \\
\hline
\end{longtable}
\caption{Caso de Teste 13 - Parser de JSON com Dados Absurdos}
\label{fig_ct27_parser_json_dados_absurdos}
\end{figure}

\begin{figure}[H]
    \centering
\begin{longtable}{|p{0.2\textwidth}|p{0.7\textwidth}|}
\hline
\multicolumn{2}{|l|}{\textbf{CT14 - Aplicação de Filtro de Média Móvel}} \\
\hline
\textbf{Tipo} & Unidade \\
\hline
\textbf{Objetivo} & Verificar aplicação correta de filtro. \\
\hline
\textbf{Pré-condições} & Função de filtro implementada.  Dados de teste com ruído disponíveis.  \\
\hline
\textbf{Procedimentos} & Rodar função de filtro com dados de teste (com ruído).  Comparar saída com valores esperados (média móvel conhecida).  \\
\hline
\textbf{Resultado esperado} & Dados suavizados sem distorção indevida. \\
\hline
\textbf{Especificação do Reparo} & Ajustar janela ou lógica de média móvel.  Corrigir cálculos que gerem resultados errados. \\
\hline
\textbf{Resultado após Reparo} & Filtro suaviza corretamente os dados e remove ruídos. \\
\hline
\end{longtable}
\caption{Caso de Teste 14 - Aplicação de Filtro de Média Móvel}
\label{fig_ct11_filtro_media_movel}
\end{figure}

\begin{figure}[H]
    \centering
\begin{longtable}{|p{0.2\textwidth}|p{0.7\textwidth}|}
\hline
\multicolumn{2}{|l|}{\textbf{CT15 - Exportação de Dados para JSON}} \\
\hline
\textbf{Tipo} & Unidade \\
\hline
\textbf{Objetivo} & Validar formatação e consistência do JSON. \\
\hline
\textbf{Pré-condições} & Função de exportação implementada e em funcionamento.  Dados em memória para exportação. \\
\hline
\textbf{Procedimentos} & Rodar função de exportação com dados em memória.  Abrir JSON e verificar cabeçalhos, formato e integridade dos dados. \\
\hline
\textbf{Resultado esperado} & Arquivo JSON gerado sem erros. \\
\hline
\textbf{Especificação do Reparo} & Corrigir escrita do arquivo JSON (formato e separadores).  Ajustar cabeçalhos e ordem de dados. \\
\hline
\textbf{Resultado após Reparo} & JSON exportado corretamente e sem inconsistências. \\
\hline
\end{longtable}
\caption{Caso de Teste 15 - Exportação de Dados para JSON}
\label{fig_ct12_exportacao_dados_csv}
\end{figure}

\begin{figure}[H]
    \centering
\begin{longtable}{|p{0.2\textwidth}|p{0.7\textwidth}|}
\hline
\multicolumn{2}{|l|}{\textbf{CT16 - Geração de gráfico}} \\
\hline
\textbf{Tipo} & Unidade \\
\hline
\textbf{Objetivo} & Validar coerência visual dos gráficos gerados. \\
\hline
\textbf{Pré-condições} & Função de geração de gráficos implementada.  Dados de teste consistentes.  \\
\hline
\textbf{Procedimentos} & Rodar função de geração de gráficos com dados de teste.  Conferir se gráfico está correto visualmente e numericamente.  \\
\hline
\textbf{Resultado esperado} & Gráfico gerado com dados coerentes e layout adequado. \\
\hline
\textbf{Especificação do Reparo} &  Ajustar lógica de plotagem (eixos e escalas).  Corrigir erros de indexação de dados.  \\
\hline
\textbf{Resultado após Reparo} & Gráficos exibem dados de forma coerente e confiável. \\
\hline
\end{longtable}
\caption{Caso de Teste 16 - Geração de gráfico}
\label{fig_ct13_geracao_grafico}
\end{figure}

\begin{figure}[H]
    \centering
\begin{longtable}{|p{0.2\textwidth}|p{0.7\textwidth}|}
\hline
\multicolumn{2}{|l|}{\textbf{CT17 - Validação do Sensor MPX5700 (Pressão)}} \\
\hline
\textbf{Tipo} & Unidade (hardware/embarcado) \\
\hline
\textbf{Objetivo} & Garantir leitura coerente da pressão durante carregamento. \\
\hline
\textbf{Pré-condições} & Sistema embarcado ligado e calibrado. \\
\hline
\textbf{Procedimentos} &  Pressurizar gradualmente a câmara.  Ler valores do sensor de pressão.  Comparar com medidor externo confiável.  \\
\hline
\textbf{Resultado esperado} & Leituras consistentes, sem desvios bruscos. \\
\hline
\textbf{Especificação do Reparo} & Calibrar sensor ou corrigir conversão no firmware. \\
\hline
\textbf{Resultado após Reparo} & Leituras de pressão confiáveis e coerentes. \\
\hline
\end{longtable}
\caption{Caso de Teste 17 - Validação do Sensor MPX5700 (Pressão)}
\label{fig_ct14_validacao_sensor_mpx5700}
\end{figure}

\begin{figure}[H]
    \centering
\begin{longtable}{|p{0.2\textwidth}|p{0.7\textwidth}|}
\hline
\multicolumn{2}{|l|}{\textbf{CT18 - Validação do Sensor MPU-6500 (Acelerômetro)}} \\
\hline
\textbf{Tipo} & Unidade (hardware/embarcado) \\
\hline
\textbf{Objetivo} & Verificar se a leitura de aceleração está coerente com o movimento. \\
\hline
\textbf{Pré-condições} & Sistema embarcado funcional. \\
\hline
\textbf{Procedimentos} &  Simular movimentos suaves e bruscos do foguete.  Observar leituras de aceleração.  \\
\hline
\textbf{Resultado esperado} & Dados refletem as variações reais de movimento. \\
\hline
\textbf{Especificação do Reparo} & Calibrar sensor e revisar código de leitura. \\
\hline
\textbf{Resultado após Reparo} & Sensor reporta acelerações reais de forma confiável. \\
\hline
\end{longtable}
\caption{Caso de Teste 18 - Validação do Sensor MPU-6500 (Acelerômetro)}
\label{fig_ct15_validacao_sensor_mpu6500}
\end{figure}

\begin{figure}[H]
    \centering
\begin{longtable}{|p{0.2\textwidth}|p{0.7\textwidth}|}
\hline
\multicolumn{2}{|l|}{\textbf{CT19 - Leitura do Sensor de Massa}} \\
\hline
\textbf{Tipo} & Unidade \\
\hline
\textbf{Objetivo} & Garantir leitura correta da massa do foguete antes do lançamento. \\
\hline
\textbf{Pré-condições} &  Sensor de massa calibrado.  Massa real conhecida.  \\
\hline
\textbf{Procedimentos} &  Colocar o foguete com massa conhecida na plataforma.  Ler os dados da célula de carga via firmware.  Comparar leitura com massa real medida em balança.  \\
\hline
\textbf{Resultado esperado} & Erro máximo aceitável ($\pm 10$g). \\
\hline
\textbf{Especificação do Reparo} & Verificar calibração e conexão do sensor. Corrigir leituras incorretas no firmware. \\
\hline
\textbf{Resultado após Reparo} & Leitura de massa consistente com o valor real. \\
\hline
\end{longtable}
\caption{Caso de Teste 19 - Leitura do Sensor de Massa}
\label{fig_ct16_leitura_sensor_massa}
\end{figure}

\begin{figure}[H]
    \centering
\begin{longtable}{|p{0.2\textwidth}|p{0.7\textwidth}|}
\hline
\multicolumn{2}{|l|}{\textbf{CT20 - Validação do Dimensionamento Energético}} \\
\hline
\textbf{Tipo} & Unidade \\
\hline
\textbf{Objetivo} & Verificar se o fornecimento de energia atende ao consumo calculado com margem de segurança. \\
\hline
\textbf{Pré-condições} &  Sistema embarcado montado com todos os sensores e atuadores conectados.  Fonte de alimentação dimensionada.  \\
\hline
\textbf{Procedimentos} &  Ligar o sistema completo.  Monitorar tensão e corrente durante 30 segundos.  Verificar se a tensão permanece estável.  Verificar se a corrente total está dentro da faixa de consumo.  \\
\hline
\textbf{Resultado esperado} & Sistema opera sem quedas de tensão ou falhas de alimentação. \\
\hline
\textbf{Especificação do Reparo} & Substituir fonte de alimentação ou ajustar conexões elétricas. \\
\hline
\textbf{Resultado após Reparo} & Sistema estável durante todo o tempo de operação. \\
\hline
\end{longtable}
\caption{Caso de Teste 20 - Validação do Dimensionamento Energético}
\label{fig_ct17_validacao_dimensionamento_energetico}
\end{figure}

\begin{figure}[H]
    \centering
\begin{longtable}{|p{0.2\textwidth}|p{0.7\textwidth}|}
\hline
\multicolumn{2}{|l|}{\textbf{CT21 - Validação do Atuador - Válvula Solenóide}} \\
\hline
\textbf{Tipo} & Unidade \\
\hline
\textbf{Objetivo} & Verificar se a válvula solenóide abre/fecha corretamente sob comando. \\
\hline
\textbf{Pré-condições} &  Sistema embarcado funcional.  Compressor e pressão dentro do esperado.  \\
\hline
\textbf{Procedimentos} &  Acionar a válvula via relé (comando direto no firmware).  Observar se há liberação/fechamento imediato de ar.  \\
\hline
\textbf{Resultado esperado} & Válvula responde rapidamente ao comando sem falhas. \\
\hline
\textbf{Especificação do Reparo} & Verificar conexões elétricas e integridade do relé. \\
\hline
\textbf{Resultado após Reparo} & Válvula abre/fecha conforme esperado. \\
\hline
\end{longtable}
\caption{Caso de Teste 21 - Validação do Atuador - Válvula Solenóide}
\label{fig_ct18_validacao_atuador_valvula_solenoide}
\end{figure}

\begin{figure}[H]
    \centering
\begin{longtable}{|p{0.2\textwidth}|p{0.7\textwidth}|}
\hline
\multicolumn{2}{|l|}{\textbf{CT22 - Validação do Atuador - Compressor}} \\
\hline
\textbf{Tipo} & Unidade \\
\hline
\textbf{Objetivo} & Verificar funcionamento do compressor 12 V durante a fase de pressurização. \\
\hline
\textbf{Pré-condições} & Sistema montado e conectado ao relé. \\
\hline
\textbf{Procedimentos} &  Enviar comando para ativar compressor.  Observar operação estável e sem superaquecimento.  \\
\hline
\textbf{Resultado esperado} & Compressor enche a câmara e atinge pressão esperada. \\
\hline
\textbf{Especificação do Reparo} & Verificar tensão de alimentação, relé e conexões físicas. \\
\hline
\textbf{Resultado após Reparo} & Compressor opera normalmente sem falhas. \\
\hline
\end{longtable}
\caption{Caso de Teste 22 - Validação do Atuador - Compressor}
\label{fig_ct19_validacao_atuador_compressor}
\end{figure}

\begin{figure}[H]
    \centering
\begin{longtable}{|p{0.2\textwidth}|p{0.7\textwidth}|}
\hline
\multicolumn{2}{|l|}{\textbf{CT23 - Validação do Atuador - Buzzer}} \\
\hline
\textbf{Tipo} & Unidade \\
\hline
\textbf{Objetivo} & Verificar se o buzzer emite contagem regressiva sonora. \\
\hline
\textbf{Pré-condições} & Firmware com contagem regressiva implementada. \\
\hline
\textbf{Procedimentos} &  Acionar comando de contagem regressiva.  Observar sequência de sons de 10 a 0.  \\
\hline
\textbf{Resultado esperado} & Sons nítidos e no tempo correto. \\
\hline
\textbf{Especificação do Reparo} & Verificar ligação do buzzer e ajuste do firmware. \\
\hline
\textbf{Resultado após Reparo} & Contagem sonora clara e funcional. \\
\hline
\end{longtable}
\caption{Caso de Teste 23 - Validação do Atuador - Buzzer}
\label{fig_ct20_validacao_atuador_buzzer}
\end{figure}

\begin{figure}[H]
    \centering
\begin{longtable}{|p{0.2\textwidth}|p{0.7\textwidth}|}
\hline
\multicolumn{2}{|l|}{\textbf{CT24 - Validação da Conectividade LoRa (RFM95W)}} \\
\hline
\textbf{Tipo} & Unidade \\
\hline
\textbf{Objetivo} & Garantir comunicação confiável entre ESP32 da base e do foguete. \\
\hline
\textbf{Pré-condições} & Ambientes de teste prontos (foguete e base). \\
\hline
\textbf{Procedimentos} &  Ligar ambos os dispositivos.  Iniciar transmissão de dados de teste do foguete para a base.  Observar recepção sem perdas significativas.  \\
\hline
\textbf{Resultado esperado} & Dados JSON recebidos sem interrupções relevantes. \\
\hline
\textbf{Especificação do Reparo} & Ajustar parâmetros de transmissão ou antenas. \\
\hline
\textbf{Resultado após Reparo} & Comunicação estável e confiável via LoRa. \\
\hline
\end{longtable}
\caption{Caso de Teste 24 - Validação da Conectividade LoRa (RFM95W)}
\label{fig_ct21_validacao_conectividade_lora}
\end{figure}

\begin{figure}[H]
    \centering
\begin{longtable}{|p{0.2\textwidth}|p{0.7\textwidth}|}
\hline
\multicolumn{2}{|l|}{\textbf{CT25 - Validação do Módulo de Armazenamento MicroSD}} \\
\hline
\textbf{Tipo} & Unidade \\
\hline
\textbf{Objetivo} & Garantir que os dados de voo são salvos corretamente no cartão. \\
\hline
\textbf{Pré-condições} & Sistema do foguete funcional e MicroSD inserido. \\
\hline
\textbf{Procedimentos} &  Iniciar simulação de voo.  Salvar dados de sensores no MicroSD.  Remover cartão e abrir no computador.  \\
\hline
\textbf{Resultado esperado} & Arquivo JSON salvo e legível. \\
\hline
\textbf{Especificação do Reparo} & Verificar comandos de escrita no firmware e integridade do MicroSD. \\
\hline
\textbf{Resultado após Reparo} & Dados salvos corretamente e legíveis. \\
\hline
\end{longtable}
\caption{Caso de Teste 25 - Validação do Módulo de Armazenamento MicroSD}
\label{fig_ct22_validacao_modulo_armazenamento_microsd}
\end{figure}