\section{Experimentos da estrutura}

Os experimentos estruturais foram realizados para validar três aspectos críticos: resistência dos materiais, estabilidade aerodinâmica e desempenho balístico. Adotou-se metodologia quantitativa com replicações para garantir confiabilidade dos resultados.

\subsection{Hipóteses levantadas}
\begin{itemize}
    \item \textbf{H1}: A garrafa PET convencional apresentará melhor desempenho ao impacto que a retornável
    \item \textbf{H2}: A geometria das aletas garantirá estabilidade com margem $\geq 1$ cal
    \item \textbf{H3}: Combinações água/pressão específicas atingirão os alvos de 10m, 20m e 30m com erro $\leq 10\%$
    \item \textbf{H4}: O mecanismo de liberação funcionará consistentemente na faixa de 1-2 bar
\end{itemize}

\subsection{Condições de contorno}
\begin{itemize}
    \item Ambiente: Campo aberto com piso gramado (coeficiente de restituição: 0.3)
    \item Instrumentação: Câmera de alta velocidade (60 fps), manômetro digital ($\pm$0.05 bar), trena laser ($\pm$0.5 cm)
    \item Replicações: 5 testes por configuração
    \item Controles: Ângulo fixo em 45°, massa constante do foguete (com lastro simulado)
\end{itemize}

\subsection{Metodologia experimental}
\subsubsection{Testes de impacto}
\begin{itemize}
    \item Protocolo: Quedas livres de 5m com 3 orientações (ponta/base/lateral)
    \item Métricas: Deformação residual (paquímetro digital $\pm$0.02 mm), inspeção visual de fraturas
    \item Amostras: 10 garrafas PET (5 convencionais, 5 retornáveis)
\end{itemize}

\subsubsection{Validação aerodinâmica}
\begin{itemize}
    \item Rastreamento de marcadores na fuselagem (câmera de alta velocidade)
    \item Cálculo do ângulo de ataque instantâneo durante fase propulsiva
    \item Análise de estabilidade pós-impacto das aletas
\end{itemize}

\subsubsection{Otimização balística}
\begin{itemize}
    \item Matriz experimental: 
    \begin{table}[H]
        \centering
        \begin{tabular}{|c|c|c|}
            \hline
            Meta (m) & Água (g) & Pressão (bar) \\
            \hline
            10 & 100 & 1.0 \\
            20 & 150 & 1.5 \\
            30 & 200 & 2.0 \\
            \hline
        \end{tabular}
    \end{table}
    \item Métricas: Alcance real (trena laser), pressão efetiva no lançamento
\end{itemize}

\subsubsection{Confiabilidade do mecanismo}
\begin{itemize}
    \item Teste de ciclo contínuo: 20 ativações em pressões crescentes (0.5-2.5 bar)
    \item Métricas: Tempo de resposta (osciloscópio), consistência do deslocamento
\end{itemize}

\subsection{Resultados}
\subsubsection{Desempenho estrutural}
\begin{itemize}
    \item PET convencional: Deformação máxima 3.2 mm sem fratura após 5 impactos
    \item PET retornável: Fratura frágil no 3º impacto (orientação lateral)
    \item Eficácia do amortecedor: Redução de 40% na aceleração de impacto medida
\end{itemize}

\subsubsection{Estabilidade aerodinâmica}
\begin{table}[H]
    \centering
    \caption{Desvios angulares médios na fase propulsiva (n=15)}
    \begin{tabular}{|c|c|c|}
        \hline
        Alcance (m) & $\bar{\theta}$ ($^\circ$) & $\sigma_\theta$ ($^\circ$) \\
        \hline
        10 & 3.2 & 0.8 \\
        20 & 2.7 & 0.6 \\
        30 & 4.1 & 1.2 \\
        \hline
    \end{tabular}
\end{table}
\begin{itemize}
    \item Zero falhas em aletas após 15 lançamentos
\end{itemize}

\subsubsection{Desempenho balístico}
\begin{table}[H]
    \centering
    \caption{Resultados de alcance (n=5 por configuração)}
    \begin{tabular}{|c|c|c|c|}
        \hline
        Alvo (m) & Real (m) & Erro (\%) & $\sigma$ (m) \\
        \hline
        10 & 9.3 & 7.0 & 0.4 \\
        20 & 19.1 & 4.5 & 0.6 \\
        30 & 29.6 & 1.3 & 0.3 \\
        \hline
    \end{tabular}
\end{table}
\textit{Nota: Pressões efetivas 0.9 bar (10m), 1.4 bar (20m), 1.9 bar (30m)}

\subsubsection{Confiabilidade do sistema}
\begin{itemize}
    \item 100\% de sucesso na faixa 1.0-2.0 bar
    \item Tempo de resposta: 0.8s $\pm$ 0.1s
    \item Deslocamento do gatilho: 12.5mm $\pm$ 0.3mm
\end{itemize}

\subsection{Análise crítica}
\begin{itemize}
    \item \textbf{H1 validada}: PET convencional demonstrou superior tenacidade
    \item \textbf{H2 validada}: Desvios angulares $<5^\circ$ comprovam estabilidade
    \item \textbf{H3 parcial}: Necessário ajuste fino para 10m (lubrificante WD-40)
    \item \textbf{H4 superada}: Mecanismo operou além da faixa especificada
    \item Limitações identificadas:
    \begin{itemize}
        \item Sensibilidade a ventos transversais (>3 m/s)
        \item Degradação progressiva do vedante após 15 ciclos
    \end{itemize}
\end{itemize}

\subsection{Conclusão experimental}
A estrutura atendeu aos requisitos fundamentais com margens seguras. As soluções derivadas dos testes - lubrificação para baixas pressões, otimização do lastro e reforço do vedante - foram incorporadas ao design final. A metodologia experimental demonstrou eficácia na validação de parâmetros críticos para as três configurações-alvo.
