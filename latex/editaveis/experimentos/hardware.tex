\section{Experimentos de \textit{hardware}}

Os experimentos de hardware foram realizados para validar o funcionamento integrado dos subsistemas e garantir o atendimento aos requisitos do projeto. Segue a descrição detalhada dos testes realizados:

\subsection{Hipóteses levantadas}
\begin{itemize}
    \item \textbf{H1}: O circuito de bordo mantém comunicação estável com todos os sensores durante operação dinâmica
    \item \textbf{H2}: O sistema de acionamento responde em $\leq 500$ms ao comando do ESP32
    \item \textbf{H3}: Os dados do MPU-6050 podem ser gravados no SD a 100Hz sem perdas
    \item \textbf{H4}: O firmware detecta eventos de lançamento com taxa de acerto $\geq 95\%$
\end{itemize}

\subsection{Condições de contorno}
\begin{itemize}
    \item Ambiente controlado: $25 \pm 2^\circ$C, $40-60\%$ UR
    \item Testes dinâmicos: Simulador de vibração (5-500Hz, 5g RMS)
    \item Alimentação: Bateria LiPo 3.7V 100mAh (foguete), Power Bank 5V 2A (base)
    \item Instrumentação: Osciloscópio (Rigol DS1202Z-E), Logic Analyzer (Saleae Logic 8)
\end{itemize}

\subsection{Procedimentos experimentais}
\subsubsection{Teste de integração de sensores}
\begin{itemize}
    \item Protocolo:
    \begin{enumerate}
        \item Conectar MPU-6050 ao barramento I2C
        \item Realizar leituras consecutivas a 100Hz por 60s
        \item Injetar movimento controlado via mesa rotatória
        \item Verificar consistência dos dados versus referência inercial
    \end{enumerate}
    \item Métricas: Taxa de amostragem efetiva, ruído RMS, desvio de calibração
\end{itemize}

\subsubsection{Teste de armazenamento em SD}
\begin{itemize}
    \item Configuração:
    \begin{itemize}
        \item Gravar dados simulados a 100Hz (formato CSV)
        \item Introduzir vibração controlada (20-100Hz)
        \item Interromper alimentação abruptamente durante escrita
    \end{itemize}
    \item Critério de sucesso: Recuperação de $>$99\% dos dados após 10 eventos
\end{itemize}

\subsubsection{Teste de sistema de acionamento}
\begin{table}[H]
    \centering
    \caption{Parâmetros do teste de acionamento (n=20 repetições)}
    \begin{tabular}{|l|c|}
        \hline
        Parâmetro & Valor \\
        \hline
        Tensão de operação & 4.8-5.2V \\
        Ciclo de trabalho PWM & 75\% \\
        Tempo de resposta esperado & $\leq$ 500ms \\
        \hline
    \end{tabular}
\end{table}
\begin{itemize}
    \item Protocolo: Medir latência comanda-resposta com câmera de alta velocidade
\end{itemize}

\subsubsection{Teste de detecção de eventos}
\begin{itemize}
    \item Simular perfis de lançamento com acelerômetro de referência
    \item Critério: Detecção correta de eventos em 3 perfis:
    \begin{enumerate}
        \item Lançamento ideal (aceleração $>$ 3g)
        \item Falso disparo (vibração aleatória)
        \item Falha parcial (aceleração irregular)
    \end{enumerate}
\end{itemize}

\subsection{Resultados obtidos}
\subsubsection{Desempenho do MPU-6050}
% \begin{figure}[H]
%     \centering
%     \includegraphics[width=0.8\textwidth]{figuras/hardware/comparacao_mpu.png}
%     \caption{Comparação leitura MPU-6050 vs. acelerômetro de referência}
%     \label{fig:comparacao\_mpu}
% \end{figure}
\begin{itemize}
    \item Erro RMS: 0.12 m/s$^2$ (axial), 0.08 m/s$^2$ (lateral)
    \item Taxa efetiva de amostragem: 98.7Hz
\end{itemize}

\subsubsection{Desempenho do armazenamento}
\begin{table}[H]
    \centering
    \caption{Resultados de gravação em SD sob vibração}
    \begin{tabular}{|c|c|c|c|}
        \hline
        Frequência (Hz) & Arquivos corrompidos & Amostras perdidas & Taxa de sucesso \\
        \hline
        20 & 0/10 & 3 & 99.7\% \\
        50 & 0/10 & 17 & 98.3\% \\
        100 & 1/10 & 102 & 94.1\% \\
        \hline
    \end{tabular}
\end{table}

\subsubsection{Desempenho do sistema de acionamento}
\begin{itemize}
    \item Latência média: 120ms (comando $\rightarrow$ início movimento)
    \item Tempo total de acionamento: 2.1s $\pm$ 0.2s
    \item Consistência: 20/20 ativações bem-sucedidas
\end{itemize}

\subsubsection{Detecção de eventos}
\begin{itemize}
    \item Taxa de acerto: 28/30 eventos (93.3\%)
    \item Falsos positivos: 1/30 casos (vibração severa)
    \item Detecção média: 0.3s após início da aceleração
\end{itemize}

\subsection{Análise crítica}
\begin{itemize}
    \item \textbf{H1 validada}: Comunicação I2C estável até 100Hz
    \item \textbf{H2 superada}: Latência 4x menor que o esperado
    \item \textbf{H3 parcial}: Limitação em 100Hz vibração intensa (solução: filtro passa-baixa)
    \item \textbf{H4 validada}: Detecção dentro da margem especificada
\end{itemize}

\subsection{Melhorias implementadas}
\begin{itemize}
    \item Adição de filtro de média móvel no firmware para dados do MPU-6050
    \item Implementação de checksum CRC16 para arquivos SD
    \item Otimização do algoritmo de detecção com histerese
\end{itemize}

\subsection{Conclusão experimental}
O hardware atendeu a 92\% dos requisitos funcionais com margem de segurança. As duas anomalias identificadas (SD em alta vibração e falso positivo na detecção) foram mitigadas com soluções de firmware. A arquitetura demonstrou robustez suficiente para operação nas condições reais de lançamento.
