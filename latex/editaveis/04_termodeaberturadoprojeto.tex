\chapter{Termo de Abertura do Projeto}

%\textcolor{red}{Termo de abertura do projeto / Project Charter. Um documento publicado pelo iniciador ou patrocinador do projeto que autoriza formalmente a existência de um projeto e fornece ao gerente do projeto a autoridade para aplicar os recursos organizacionais nas atividades do projeto.}

\section{Dados do projeto}
\begin{description}
    \item [Nome do Projeto:] Foguete d’Água com Base Automatizada
    \item [Data de abertura:] 23/04/2025
    \item [Código:] 2-A
    \item [Patrocinador:] Universidade de Brasília
    % \item [Responsável:] ???
    \item[Gerente do projeto:] Vitor Feijó Leonardo \\
    Matrícula: 221008516 \\
    E-mail: \texttt{221008516@aluno.unb.br} \\
    Telefone: +55\,(61)\,99243-6348
\end{description}

\section{Objetivos}

Desenvolver, até 16 de julho de 2025, um sistema original e funcional de lançamento automatizado de foguetes d'água, com controle de trajetória baseado em dados reais (pressão, ângulo, altitude, velocidade, aceleração), assegurando a reutilização em três lançamentos com precisão de até ±0,5 metro nas distâncias de 10 m, 20 m e 30 m, respeitando o limite orçamentário de R\$ 1.000{,}00.


\section{Mercado-alvo}

O projeto tem como mercado-alvo principal instituições educacionais e grupos de pesquisa que atuam nas áreas de engenharia, física aplicada e educação científica, especialmente em níveis médio, técnico e superior. Entre os potenciais beneficiários estão professores e estudantes envolvidos em atividades práticas de ensino de ciências, clubes de astronomia e olimpíadas científicas, que podem utilizar o sistema como ferramenta didática para explorar conceitos de física, matemática, programação e controle de sistemas dinâmicos.

Além do ambiente acadêmico, o projeto atende também às necessidades de pesquisadores e entusiastas da engenharia experimental que necessitam de plataformas acessíveis para testes de controle de trajetória, aquisição de dados em tempo real e automação de sistemas mecatrônicos. A proposta oferece um modelo de baixo custo e alta reusabilidade, adequado para experimentação segura em ambientes controlados.

De forma indireta, o sistema pode ainda inspirar iniciativas voltadas à popularização da ciência e inovação tecnológica, sendo compatível com projetos de extensão universitária e feiras científicas, contribuindo para a disseminação do conhecimento e estímulo à formação científica em contextos educacionais diversos.

\section{Requisitos}

%\subsection{Requisitos Gerais}
\begin{table}[htpb]
\centering
\scriptsize
\setlength{\tabcolsep}{4pt}
\caption{Requisitos Gerais}
\begin{tabular}{|c|p{0.85\linewidth}|}
\hline
\textbf{Código} & \textbf{Descrição} \\
\hline
RF1 & O sistema deve executar três lançamentos reutilizáveis de um mesmo foguete d'água, programados para alcançarem 10\,m, 20\,m e 30\,m, com tolerância de ±0,5\,m. \\
\hline
RF2 & O software embarcado deve coletar, em tempo real, sinais de volume de água, pressão interna, ângulo de lançamento, posição e altitude, velocidade e aceleração (conforme especificações de precisão). \\
\hline
RF3 & O firmware deve armazenar os dados em um MicroSD e permitir a recuperação posterior para análise. \\
\hline
RF4 & O módulo de processamento principal deve armazenar em arquivo JSON, garantindo integridade no histórico de cada voo. \\
\hline
\end{tabular}
\label{tab:requisitos-funcionais}
\end{table}


\begin{table}[htpb]
\centering
\scriptsize % Font size as per your example
\setlength{\tabcolsep}{4pt} % Column separation as per your example
\caption{Requisitos de Software} % A suitable caption for this table
%\begin{tabular}{|l|p{10cm}|} % Defines columns: Left-aligned ID, Paragraph (10cm width) for Description
\begin{tabular}{|c|p{0.85\linewidth}|}
%\hline
%\multicolumn{2}{|c|}{\textbf{REQUISITOS}} \\ % Main header spanning both columns
\hline
\textbf{Código} & \textbf{Descrição} \\
\hline
RQ01 & Exibir gráfico de velocidade vertical vs. tempo. \\
\hline
RQ02 & Exibir gráfico de aceleração vertical vs. tempo. \\
\hline
RQ03 & Exibir gráfico de dispersão da trajetória no plano X e Y. \\
\hline
RQ04 & Exibir gráfico de altitude vs. tempo. \\
\hline
RQ05 & Exibir valores máximos e mínimos de aceleração, velocidade e ângulo. \\
\hline
RQ06 & Exibir tempo de execução e intervalos de amostragem. \\
\hline
RQ07 & Exportar dados do voo em JSON. \\
\hline
RQ08 & Aplicar filtro de média móvel nos dados de voo. \\
\hline
\end{tabular}
\label{tab:requisitos_detalhados} % A label for referencing the table
\end{table}


%\subsection{Requisitos de Hardware}
\begin{table}[htpb]
\centering
\scriptsize
\setlength{\tabcolsep}{4pt}
\caption{Requisitos de Hardware}
\begin{tabular}{|c|p{0.85\linewidth}|}
\hline
\textbf{Código} & \textbf{Descrição} \\
\hline
RH1 & Microcontrolador com ADC de 12 bits ou superior, memória RAM maior que 64 kB e Flash maior que 256 kB. \\
\hline
RH2 & Bateria capaz de alimentar todo o sistema. \\
\hline
\end{tabular}
\label{tab:requisitos-hardware}
\end{table}

\begin{samepage}
%\subsection{Requisitos de Custos e Materiais}
\begin{table}[htpb]
\centering
\scriptsize
\setlength{\tabcolsep}{4pt}
\caption{Requisitos de Custos e Materiais}
\begin{tabular}{|c|p{0.85\linewidth}|}
\hline
\textbf{Código} & \textbf{Descrição} \\
\hline
RC1 & Custo total de componentes eletrônicos, mecânicos e estruturais não deve exceder R\$ 1.000,00. \\
\hline
\end{tabular}
\label{tab:requisitos-custos}
\end{table}
\end{samepage}

\section{Justificativa}

O projeto justifica-se pela necessidade de promover a aplicação prática e integrada de conhecimentos das diferentes engenharias da Faculdade de Ciências e Tencologias em Engenharia (FCTE), por meio do desenvolvimento de uma solução original, segura e funcional para o lançamento automatizado de foguetes d’água. 

Além disso, o projeto representa uma oportunidade para suprir a carência de sistemas didáticos acessíveis e reutilizáveis voltados ao ensino de física aplicada, automação e controle de trajetória em ambientes educacionais. Desse modo, integrando sensores e ferramentas de análise, a solução contribui para o avanço de práticas experimentais no ensino de engenharia e ciências, podendo ser adotada por instituições de ensino e centros de pesquisa como recurso pedagógico. 



\section{Indicadores}

\begin{enumerate}
  \item \textbf{Precisão da distância atingida:} diferença absoluta entre a distância real alcançada e as metas de 10m, 20m e 30m, com tolerância máxima de 0,5m.
  \item \textbf{Número de lançamentos bem-sucedidos:} total de disparos realizados com o mesmo foguete sem falhas estruturais ou funcionais (meta: 3 lançamentos).
  \item \textbf{Tempo de montagem e preparação para o lançamento:} tempo médio entre o início da montagem até a execução do disparo (em minutos).
  \item \textbf{Número de dados coletados por voo:} total de amostras válidas registradas por sensores durante cada lançamento (em $N$ leituras).
  \item \textbf{Taxa de perda de dados:} percentual de dados esperados que não foram coletados ou foram invalidados por falha de sensor, ou transmissão.
  \item \textbf{Tempo total de voo:} duração média entre o lançamento e o retorno do foguete ao solo, medido por sensores (em segundos).
  \item \textbf{Desvio angular no lançamento:} diferença entre o ângulo configurado e o ângulo real de disparo detectado (em graus).
  \item \textbf{Número de erros críticos no sistema embarcado:} quantidade de falhas que impedem a coleta, transmissão ou registro de dados em um lançamento.
  \item \textbf{Número de funcionalidades do software implementadas com sucesso:} total de recursos entregues em relação ao planejado (ex.: gráficos, filtro de ruído, exportação de dados, visualização histórica, etc.).
  \item \textbf{Atraso no cronograma:} dias de diferença entre o plano de entregas e a data real de finalização de cada etapa do projeto (em dias corridos).
\end{enumerate}
