\chapter{Lições aprendidas}

% \textcolor{red}{Documentar as lições aprendidas de modo a aperfeiçoar os processos e evitar que os erros e problemas encontrados se repitam em futuros projetos.}

As principais experiências adquiridas durante o desenvolvimento do projeto, identificando práticas bem-sucedidas e desafios encontrados. Serve para aprimorar processos futuros e prevenir a repetição de erros, organizando-se em: análise SWOT, avaliação de objetivos, prazos, orçamento e escopo, além de recomendações para projetos futuros.

\section{Análise SWOT}

% \textcolor{red}{A análise SWOT (\textit{Strengths}, \textit{Weaknesses}, \textit{Opportunities}, \textit{Threats}) é usada para identificar os pontos fortes e fracos do produto, e as principais oportunidades e ameaças normalmente associadas aos pontos fortes e fracos identificados. Preencha a Tabela \ref{tab:swot} de forma correspondente.}

A Tabela \ref{tab:swot} sintetiza a avaliação estratégica do projeto, categorizando fatores determinantes para o sucesso e riscos críticos:

\begin{table}[H]
\centering
\caption{Análise SWOT do produto.}
\begin{tabular}{|c|c|c|} \hline
 & \textbf{Pontos fortes} & \textbf{Pontos fracos}  \\ \hline 
    {\parbox{0.1\textwidth}{ \textbf{Fatores \\ internos}}}  &
    {\parbox{0.4\textwidth}{ Forças: 
        \\
		%\textcolor{red}{Descreva seus diferenciais competitivos: aquilo que faz de você melhor que seus concorrentes.}
        \begin{itemize}
            \item Conhecimento em sistemas embarcados e montagem de drones
            \item Trabalho em equipe e colaboração
			\item Organização e compartilhamento de conhecimento
			\item Bom conhecimento em energia
			\item Esforço e participação
			\item Facilidade na codificação
        \end{itemize} }} & 
    {\parbox{0.4\textwidth}{ Fraquezas:
        \\
		% \textcolor{red}{Descreva suas principais deficiências: aquilo que faz de você pior que seus concorrentes, e pontos a serem melhorados ou revistos.}
        \begin{itemize}
            \item Ausência de engenheiros eletrônicos
            \item Atraso nos testes por demora de equipamentos
			\item Dificuldade de sincronização da equipe
			\item Conhecimento constrúido durante a aplicação
        \end{itemize} }}\\ \hline
    {\parbox{0.1\textwidth}{ \textbf{Fatores \\ externos}}}  &
    {\parbox{0.4\textwidth}{ Oportunidades:
        \\
		% \textcolor{red}{Descreva as principais oportunidades identificadas: eventos potenciais que podem gerar grandes benefícios.}
        \begin{itemize}
            \item Aprofundamento em sistemas embarcados e eletrônica
            \item Colobaração interdisciplinar futura
			\item Desenvolvimento de documentação e melhores práticas
			\item Aplicação de conhecimento em energia
        \end{itemize} }} & 
    {\parbox{0.4\textwidth}{ Ameaças:
        \\
		%\textcolor{red}{Descreva as principais ameaças identificadas: eventos potenciais que podem causar um grande estrago ao seu negócio.}
        \begin{itemize}
            \item Atrasos no projeto
            \item Erros graves de hardware
			\item Problemas inesperados de integração
			\item Desgaste da equipe
			\item Dependência de conhecimentos específicos
        \end{itemize} }}\\ \hline
\end{tabular}
\label{tab:swot}
\end{table}


\section{Planejado x realizado}
\subsection{Objetivos}

% \textcolor{red}{Os objetivos foram alcançados? Responda as questões e comente os pontos mais relevantes – até 700 caracteres, considerando os espaços.}
A visão da equipe é de que os objetivos da matéria e da equipe foram cumpridos de forma satisfatória. Essa lógica é embasada na visão de que Projeto integrador 1 seria uma forma de unir os alunos de diferentes engenharias em prol da troca de conhecimento, trabalho em equipe e construção de um produto com base em requisitos rigorosos. Nesse sentido, por termos conseguido ao final da disciplina construir o foguete e sua base automatizada consideramos um sucesso, mesmo não tendo atingido um dos alvos por erro de precisão. 

\subsection{Prazos}

%\textcolor{red}{O projeto foi entregue dentro do prazo? Responda as questões e comente os pontos mais relevantes – até 700 caracteres, considerando os espaços.}

Os prazos foram muito bem definidos com a ferramenta Gantt e todos tinha conhecimento da data limite de suas tarefas, assim como conhecimento de suas respectivas responsabilidades. Isso auxiliou muito na hora das entregas, existiram alguns atrasos, mas foram muito pífios pois nosso planejamento considerou que eram pessoas ocupadas e que deixariam de fato até o ultimo minuto pra entregarem. 


\subsection{Orçamento}

%\textcolor{red}{O projeto foi entregue dentro do orçamento? Responda as questões e comente os pontos mais relevantes – até 700 caracteres, considerando os espaços.}

O orçamento desembolsado, excluindo o honorário dos integrantes, ficou dentro do esperado sem onerar financeiramente nenhum dos integrantes. 

\subsection{Escopo}

%\textcolor{red}{O  projeto atendeu ao escopo? Responda as questões e comente os pontos mais relevantes – até 700 caracteres, considerando os espaços.}

O escopo do projeto era fazer uma base automatizada para realizar um lançamento eletromecânico, com capacidade de acertar 3 alvos com uma certa precisão, coletar dados durante os voos e ter a capacidade de montar análises com um software. Tende em vista que realizamos todos os principais requisitos deste projeto, consideramos uma atenção ao escopo notável, sem fugas ou extrapolamento de orçamento. 


\section{Projeto}


%\textcolor{red}{Comente os pontos mais relevantes a serem aperfeiçoados ou adotados em próximos projetos.}

A realização de mais testes é vital, seria um ponto de melhoria. A ausência de um aluno do curso de engenharia eletrônica e de energia foi um ponto crítico que afetou negativamente nosso desempenho como equipe nos relatórios, pois foram nossas menores notas. 

\subsection{Pontos fortes}
\begin{enumerate}
	\item Comunicação ágil através de reuniões diárias de 15min
	\item Documentação técnica detalhada com controle de versão
	\item Prototipagem rápida usando impressão 3D local
	\item Sistema de backup redundante para dados críticos
\end{enumerate}

\subsection{Pontos fracos}
\begin{enumerate}
	\item Testes insuficientes na fase inicial (apenas 43\% cobertura)
	\item Falta de plano B para componentes de entrega demorada
	\item Conhecimento em eletrônica adquirido durante projeto
	\item Subestimação do tempo para integração hardware-software
\end{enumerate}

\section{Recomendações para projetos futuros}
\begin{enumerate}
	\item Alocar 20\% do cronograma para testes exploratórios
	\item Estabelecer parcerias com laboratórios de eletrônica
	\item Implementar sistema de monitoramento contínuo de riscos
	\item Desenvolver protótipos funcionais nas 4 primeiras semanas
\end{enumerate}

\section{Questões em aberto}
\begin{enumerate}
	\item Implementar compensação térmica para o MPU-6050
	\item Otimizar algoritmo de detecção de lançamento para 95\% acurácia
	\item Reduzir consumo do SD Card em 40\% com técnicas de buffer
	\item Validar vida útil da estrutura após 50 ciclos de pressurização
\end{enumerate}

\section{Desempenho dos fornecedores}
\subsection{Fornecedores com desempenho acima do esperado}
\textbf{Leroy Merlin}: Entregas sempre antecipadas para componentes de PVC.

\subsection{Fornecedores com desempenho conforme esperado}
\textbf{AliexPress}: Compra de peças, módulos e equipamentos dentro do combinado.

\subsection{Fornecedores com desempenho abaixo do esperado}
Não foram identificados fornecedores com desempenho abaixo do esperado, todos cumpriram os prazos e especificações acordadas.



% \subsection{Pontos fortes}
% \begin{enumerate}
% 	\item \textcolor{red}{Até 200 caracteres, considerando os espaços.}
% 	\item \textcolor{red}{Até 200 caracteres, considerando os espaços.}
% 	\item \textcolor{red}{Até 200 caracteres, considerando os espaços.}
% 	\item \textcolor{red}{Até 200 caracteres, considerando os espaços.}
% \end{enumerate}
% \subsection{Pontos fracos}
% \begin{enumerate}
% 	\item \textcolor{red}{Até 200 caracteres, considerando os espaços.}
% 	\item \textcolor{red}{Até 200 caracteres, considerando os espaços.}
% 	\item \textcolor{red}{Até 200 caracteres, considerando os espaços.}
% 	\item \textcolor{red}{Até 200 caracteres, considerando os espaços.}
% \end{enumerate}

% \section{Recomendações para projetos futuros}

% \begin{enumerate}
% 	\item \textcolor{red}{Até 200 caracteres, considerando os espaços.}
% 	\item \textcolor{red}{Até 200 caracteres, considerando os espaços.}
% 	\item \textcolor{red}{Até 200 caracteres, considerando os espaços.}
% 	\item \textcolor{red}{Até 200 caracteres, considerando os espaços.}
% \end{enumerate}

% \section{Questões em aberto}

% \textcolor{red}{Usar caso haja alguma questão pendente em relação às entregas do projeto (Ex.: Requisitos não entregues, melhoria na programação do software, troca de sensores, entre outros.}

% \begin{enumerate}
% 	\item \textcolor{red}{Até 200 caracteres, considerando os espaços.}
% 	\item \textcolor{red}{Até 200 caracteres, considerando os espaços.}
% 	\item \textcolor{red}{Até 200 caracteres, considerando os espaços.}
% 	\item \textcolor{red}{Até 200 caracteres, considerando os espaços.}
% \end{enumerate}

% \section{Desempenho dos fornecedores}

% \subsection{Fornecedores com desempenho acima do esperado}
% \textcolor{red}{Fornecedores que você certamente convidaria para um próximo projeto. Citar AAAA – justificativa. Até 200 caracteres, considerando os espaços.}

% \subsection{Fornecedores com desempenho conforme esperado}
% \textcolor{red}{Fornecedores que você provavelmente convidaria para um próximo projeto. Citar AAAA – justificativa. Até 200 caracteres, considerando os espaços.}

% \subsection{Fornecedores com desempenho abaixo do esperado}
% \textcolor{red}{Fornecedores que você não convidaria para um próximo projeto. Citar AAAA – justificativa. Até 200 caracteres, considerando os espaços.}
