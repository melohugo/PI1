\chapter{Lições aprendidas}

\textcolor{red}{Documentar as lições aprendidas de modo a aperfeiçoar os processos e evitar que os erros e problemas encontrados se repitam em futuros projetos.}

\section{Análise SWOT}

\textcolor{red}{A análise SWOT (\textit{Strengths}, \textit{Weaknesses}, \textit{Opportunities}, \textit{Threats}) é usada para identificar os pontos fortes e fracos do produto, e as principais oportunidades e ameaças normalmente associadas aos pontos fortes e fracos identificados. Preencha a Tabela \ref{tab:swot} de forma correspondente.}

\begin{table}[ht]
% \label{tab:swot}
\centering
\caption{Análise SWOT do produto.}
\begin{tabular}{|c|c|c|} \hline
 & \textbf{Pontos fortes} & \textbf{Pontos fracos}  \\ \hline 
    {\parbox{0.1\textwidth}{ \textbf{Fatores \\ internos}}}  &
    {\parbox{0.4\textwidth}{ Forças: 
        \\ \textcolor{red}{Descreva seus diferenciais competitivos: aquilo que faz de você melhor que seus concorrentes.}
        \begin{itemize}
            \item Lorem ipsum
            \item Lorem ipsum
        \end{itemize} }} & 
    {\parbox{0.4\textwidth}{ Fraquezas:
        \\ \textcolor{red}{Descreva suas principais deficiências: aquilo que faz de você pior que seus concorrentes, e pontos a serem melhorados ou revistos.}
        \begin{itemize}
            \item Lorem ipsum
            \item Lorem ipsum
        \end{itemize} }}\\ \hline
    {\parbox{0.1\textwidth}{ \textbf{Fatores \\ externos}}}  &
    {\parbox{0.4\textwidth}{ Oportunidades:
        \\ \textcolor{red}{Descreva as principais oportunidades identificadas: eventos potenciais que podem gerar grandes benefícios.}
        \begin{itemize}
            \item Lorem ipsum 
            \item Lorem ipsum
        \end{itemize} }} & 
    {\parbox{0.4\textwidth}{ Ameaças:
        \\ \textcolor{red}{Descreva as principais ameaças identificadas: eventos potenciais que podem causar um grande estrago ao seu negócio.}
        \begin{itemize}
            \item Lorem ipsum 
            \item Lorem ipsum
        \end{itemize} }}\\ \hline
\end{tabular}
\label{tab:swot}
\end{table}


\section{Planejado x realizado}
\subsection{Objetivos}
\textcolor{red}{Os objetivos foram alcançados? Responda as questões e comente os pontos mais relevantes – até 700 caracteres, considerando os espaços.}
\subsection{Prazos}
\textcolor{red}{O projeto foi entregue dentro do prazo? Responda as questões e comente os pontos mais relevantes – até 700 caracteres, considerando os espaços.}
\subsection{Orçamento}
\textcolor{red}{O projeto foi entregue dentro do orçamento? Responda as questões e comente os pontos mais relevantes – até 700 caracteres, considerando os espaços.}
\subsection{Escopo}
\textcolor{red}{O  projeto atendeu ao escopo? Responda as questões e comente os pontos mais relevantes – até 700 caracteres, considerando os espaços.}


\section{Projeto}
\textcolor{red}{Comente os pontos mais relevantes a serem aperfeiçoados ou adotados em próximos projetos.}
\subsection{Pontos fortes}
\begin{enumerate}
	\item \textcolor{red}{Até 200 caracteres, considerando os espaços.}
	\item \textcolor{red}{Até 200 caracteres, considerando os espaços.}
	\item \textcolor{red}{Até 200 caracteres, considerando os espaços.}
	\item \textcolor{red}{Até 200 caracteres, considerando os espaços.}
\end{enumerate}
\subsection{Pontos fracos}
\begin{enumerate}
	\item \textcolor{red}{Até 200 caracteres, considerando os espaços.}
	\item \textcolor{red}{Até 200 caracteres, considerando os espaços.}
	\item \textcolor{red}{Até 200 caracteres, considerando os espaços.}
	\item \textcolor{red}{Até 200 caracteres, considerando os espaços.}
\end{enumerate}

\section{Recomendações para projetos futuros}

\begin{enumerate}
	\item \textcolor{red}{Até 200 caracteres, considerando os espaços.}
	\item \textcolor{red}{Até 200 caracteres, considerando os espaços.}
	\item \textcolor{red}{Até 200 caracteres, considerando os espaços.}
	\item \textcolor{red}{Até 200 caracteres, considerando os espaços.}
\end{enumerate}

\section{Questões em aberto}

\textcolor{red}{Usar caso haja alguma questão pendente em relação às entregas do projeto (Ex.: Requisitos não entregues, melhoria na programação do software, troca de sensores, entre outros.}

\begin{enumerate}
	\item \textcolor{red}{Até 200 caracteres, considerando os espaços.}
	\item \textcolor{red}{Até 200 caracteres, considerando os espaços.}
	\item \textcolor{red}{Até 200 caracteres, considerando os espaços.}
	\item \textcolor{red}{Até 200 caracteres, considerando os espaços.}
\end{enumerate}

\section{Desempenho dos fornecedores}

\subsection{Fornecedores com desempenho acima do esperado}
\textcolor{red}{Fornecedores que você certamente convidaria para um próximo projeto. Citar AAAA – justificativa. Até 200 caracteres, considerando os espaços.}

\subsection{Fornecedores com desempenho conforme esperado}
\textcolor{red}{Fornecedores que você provavelmente convidaria para um próximo projeto. Citar AAAA – justificativa. Até 200 caracteres, considerando os espaços.}

\subsection{Fornecedores com desempenho abaixo do esperado}
\textcolor{red}{Fornecedores que você não convidaria para um próximo projeto. Citar AAAA – justificativa. Até 200 caracteres, considerando os espaços.}